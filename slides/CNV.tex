\documentclass{beamer}
%
% Choose how your presentation looks.
%
% For more themes, color themes and font themes, see:
% http://deic.uab.es/~iblanes/beamer_gallery/index_by_theme.html
%
\mode<presentation>
{
  \usetheme{Madrid}      % or try Darmstadt, Madrid, Warsaw, ...
  \usecolortheme{default} % or try albatross, beaver, crane, ...
  \usefonttheme{default}  % or try serif, structurebold, ...
  \setbeamertemplate{navigation symbols}{}
  \setbeamertemplate{caption}[numbered]
}

\usepackage[english]{babel}
\usepackage[utf8x]{inputenc}
\usepackage{graphicx}
\usepackage{array}

\title[CNV]{Comunicação Não-Violenta (CNV)}
\author{Tiago F. Tavares}
\institute{FEEC -- UNICAMP}
\date{}

\begin{document}

\begin{frame}
  \titlepage
\end{frame}

% Uncomment these lines for an automatically generated outline.
%\begin{frame}{Outline}
%  \tableofcontents
%\end{frame}

\section{Introdução}

\begin{frame}[fragile]{Previously, on EA87[96]...}
  \centering
  \Large
  \begin{itemize}
    \item Compiladores: RegEx e GLC
    \item Compilação parcial e distribuição de código
  \end{itemize}
\end{frame}

\begin{frame}{Grandes empreendimentos humanos}
  \LARGE
  \begin{itemize}
    \item Google
    \item Youtube
    \item Facebook
  \end{itemize}
\end{frame}

\begin{frame}{Grandes empreendimentos humanos}
  \Large
  Por que será que alguns conflitos humanos levam a situações violentas, e
  outros são resolvidos com soluções pacíficas?
\end{frame}

\begin{frame}{Problema 1: O Mundo Que Eu Vejo}
  \Large
  Discussão: o copo que eu vejo é o mesmo copo que está aqui?
\end{frame}

\begin{frame}{Problema 1: O Mundo Que Eu Vejo}
  \Large
  O que é o copo que eu vejo? Como ele se relaciona com o copo que está aqui?
\end{frame}

\begin{frame}{Problema 2: O Mundo Que Eu Sinto}
  \Large
  Fatos objetivos? Idéias/opiniões? Sentimentos?


  Exercício 1!
\end{frame}

\begin{frame}{Problema 2: O Mundo Que Eu Sinto}
  \Large
  Exercício 2!
\end{frame}

\begin{frame}{Problema 3: O Processo De Sentir}
  \Large
  Discussão: pense em uma comida que você não gosta. Onde está o não-gostar?
\end{frame}

\begin{frame}{Problema 3: O Processo De Sentir}
  \Large
  Onde ``mora'' o não-gostar? Quem é que não-gosta de algo?

  Exercício 3
\end{frame}

\begin{frame}{Problema 4: Colocando na prática...}
  \Large
  Exercício 4
\end{frame}

\begin{frame}{Problema 5: Mas, como eu resolvo...?}
  \LARGE
  Hipótese: todos os seres humanos precisam de coisas semelhantes, e ficam
  frustrados quando suas necessidades não são atendidas. Que tipos de
  necessidades nós temos?
\end{frame}

\begin{frame}{Problema 5: Mas, como eu resolvo...?}
  \Large
  Exercício 5
\end{frame}

\begin{frame}{Problema 5: Mas, como eu resolvo...?}
  \Large
  Exercício 6
\end{frame}

\begin{frame}{Problema 6: Um plano de solução...}
  \Large
  Exercício 7
\end{frame}

\begin{frame}{Problema 7: ...e executar o plano!}
  \Large
  Exercício 8
\end{frame}


\end{document}

