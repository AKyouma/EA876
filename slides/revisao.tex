\documentclass{beamer}
%
% Choose how your presentation looks.
%
% For more themes, color themes and font themes, see:
% http://deic.uab.es/~iblanes/beamer_gallery/index_by_theme.html
%
\mode<presentation>
{
  \usetheme{Madrid}      % or try Darmstadt, Madrid, Warsaw, ...
  \usecolortheme{default} % or try albatross, beaver, crane, ...
  \usefonttheme{default}  % or try serif, structurebold, ...
  \setbeamertemplate{navigation symbols}{}
  \setbeamertemplate{caption}[numbered]
}

\usepackage[english]{babel}
\usepackage[utf8x]{inputenc}
\usepackage{graphicx}
\usepackage{array}

\title[Revisao]{Revisão}
\author{Tiago F. Tavares}
\institute{FEEC -- UNICAMP}
\date{}

\begin{document}

\begin{frame}
  \titlepage
\end{frame}

% Uncomment these lines for an automatically generated outline.
%\begin{frame}{Outline}
%  \tableofcontents
%\end{frame}

\section{Introdução}

\begin{frame}{Objetivos}
  \Large
  \begin{itemize}
    \item Lembrar todos os assuntos que foram abordados na disciplina
    \item Classificar o próprio aprendizado como níveis de taxonomia do Bloom
    \item Propor atividades de verificação de aprendizado usando a taxonomia de
      Bloom
  \end{itemize}
\end{frame}

\begin{frame}[fragile]{Sobre esta aula}
\LARGE
  FAÇA ANOTAÇÕES de tudo o que for discutido nesta aula. Ao fim da aula, tire
  uma foto de suas anotações e envie para o Tiago
  \url{tavares@dca.fee.unicamp.br}.
\end{frame}

\begin{frame}[fragile]{Sobre o curso...}
  \Large
  Junto ao seu grupo, lembre-se de todos os assuntos que foram abordados neste
  curso. Faça uma lista de todos esses assunto, da maneira mais detalhada que
  conseguir.

  \textbf{Exemplo (MC102):} ``variáveis'', ``laços'', ``condicionais''

  \textbf{Exemplo ruim (MC102):} ``linguagem C'', ``algoritmos''

  \vspace{1cm}

  Sua lista provavelmente chegará a ter entre 5 e 10 assuntos diferentes.
\end{frame}

\begin{frame}[fragile]{Taxonomia de Bloom}
  \large
  Após ler o texto sobre a Taxonomia de Bloom de objetivos educacionais, decida
  em qual nível de conhecimento você se posiciona quanto aos assuntos abaixo.
  Sua decisão deve se basear numa evidência, isto é, numa ação que você já fez:
  \begin{enumerate}
    \item Estruturas condicionais tipo \texttt{if}
    \item Integrais (de cálculo)
    \item Mecânica clássica
    \item Capacitores
    \item Arte do Século XX
    \item Entomologia
  \end{enumerate}
\end{frame}

\begin{frame}[fragile]{Taxonomia de Bloom (2)}
  \large
  Decida em que nível de conhecimento, usando a Taxonomia de Bloom, você domina
  cada um dos assuntos deste curso que elegeu anteriormente. Para cada decisão,
  use uma evidência, assim como fizemos no exercício anterior.
\end{frame}

\begin{frame}[fragile]{Análise de um exercício (1)}
  \large
  Em qual nível da Taxonomia de Bloom está o exercício abaixo?
  \vspace{1cm}
  \textit{Por que gramáticas regulares não podem ser usadas para decidir sobre o
  balanceamento de parênteses?}
\end{frame}

\begin{frame}[fragile]{Análise de um exercício (2)}
  \large
  Em qual nível da Taxonomia de Bloom está o exercício abaixo?
  \vspace{1cm}
  \textit{Quais são as condições necessárias para a ocorrência de deadlock em um
  algoritmo?}
\end{frame}

\begin{frame}[fragile]{Criar um exercício}
  \large
  Junto ao seu grupo, decida qual é o assunto mais relevante deste curso,
  dentre os que você
  marcou. Após, proponha uma pergunta que só pode ser respondida por pessoas que
  dominem aquele assunto no nível de conhecimento que você escolheu. Escreva seu
  exercício em uma folha de papel separada, junto ao assunto e ao nível de
  conhecimento desejados. Escreva também uma solução.
\end{frame}

\begin{frame}[fragile]{Revisar um exercício}
  \large
  Entregue seu exercício a um outro
  grupo para revisão. A revisão deverá verificar se (a) o assunto é relevante,
  (b) se o exercício, de fato, corresponde ao nível de conhecimento proposto
  pelo grupo, e (c) se a solução está correta e é factível.
\end{frame}

\begin{frame}[fragile]{Criar uma lista de exercícios}
  \large
  Após a revisão, escreva seu exercício na lousa. Em grupo, decidiremos por
  quais exercícios serão colocados na prova.
\end{frame}



\end{document}
