\documentclass{beamer}
%
% Choose how your presentation looks.
%
% For more themes, color themes and font themes, see:
% http://deic.uab.es/~iblanes/beamer_gallery/index_by_theme.html
%
\mode<presentation>
{
  \usetheme{Madrid}      % or try Darmstadt, Madrid, Warsaw, ...
  \usecolortheme{default} % or try albatross, beaver, crane, ...
  \usefonttheme{default}  % or try serif, structurebold, ...
  \setbeamertemplate{navigation symbols}{}
  \setbeamertemplate{caption}[numbered]
}

\usepackage[english]{babel}
\usepackage[utf8x]{inputenc}
\usepackage{graphicx}
\usepackage{array}

\title[07-Yacc]{Yacc}
\author{Tiago F. Tavares}
\institute{FEEC -- UNICAMP}
\date{}

\begin{document}

\begin{frame}
  \titlepage
\end{frame}

% Uncomment these lines for an automatically generated outline.
%\begin{frame}{Outline}
%  \tableofcontents
%\end{frame}

\section{Introdução}

\begin{frame}{Objetivos}
  \Large
  \begin{itemize}
    \item Entender o que é e como usar o Yacc.
    \item Entender a sintaxe do Yacc.
    \item Entender como compilar programas usando Yacc.
    \item Construir uma pequena aplicação usando Yacc.
    \item Entender o que é análise sintática.
  \end{itemize}
\end{frame}

\begin{frame}[fragile]{Live Coding}
  Yacc é uma ferramenta que permite programar regras de formação baseadas nos
  tokens gerados pelo Lex, e associar comportamentos específicos à aplicação de
  regras de formação específicas.
  \begin{verbatim}
  %{
      /* Codigo C
  %token (lista de tokens definidos no Lex)
  %}
     <----  definicoes
  %%
                 <--- Regras de formação
  %%
  /* Codigo C */
  \end{verbatim}
\end{frame}

\begin{frame}[fragile]{Usando yylval}
  \large
  \textsc{yylval} é um union que pode ser re-definido pelo programador e que contém o
  valor relacionado ao token que Lex retorna ao Yacc:

  \begin{verbatim}
  [0-9]+  {
         yylval = atoi(yytext); /* Valor do token */
         return INT; /* Tipo do token */
          }
  \end{verbatim}
\end{frame}

\begin{frame}[fragile]{Aplicando regras no sentido reverso}
  \large
  O que fazer quando encontrei uma regra de formação válida?
  \begin{verbatim}
  E:
     INT '+' INT { $$ = $1 + $3 }

  $$ é o valor que "sobe" a árvore sintática.
  $1, $2, $3 são os valores (yylval) dos tokens
  \end{verbatim}

\end{frame}

\begin{frame}[fragile]{Live Coding}
  \large
  Regras de formação definidas ANTES têm precedência sobre as regras definidas
  DEPOIS:
  \begin{verbatim}
    E:
       INT
       | '(' E ')'
       | E '+' E
  \end{verbatim}
  Em qual ordem as regras seriam aplicadas na expressão: \\ \textsc{2 + (5 + 1)}?
\end{frame}

\begin{frame}{Exercício}
\large
  Exercício 2
\end{frame}



\end{document}
