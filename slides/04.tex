\documentclass{beamer}
%
% Choose how your presentation looks.
%
% For more themes, color themes and font themes, see:
% http://deic.uab.es/~iblanes/beamer_gallery/index_by_theme.html
%
\mode<presentation>
{
  \usetheme{Madrid}      % or try Darmstadt, Madrid, Warsaw, ...
  \usecolortheme{default} % or try albatross, beaver, crane, ...
  \usefonttheme{default}  % or try serif, structurebold, ...
  \setbeamertemplate{navigation symbols}{}
  \setbeamertemplate{caption}[numbered]
}

\usepackage[english]{babel}
\usepackage[utf8x]{inputenc}
\usepackage{graphicx}
\usepackage{array}

\title[05-Análise Léxica]{Análise Léxica}
\author{Tiago F. Tavares}
\institute{FEEC -- UNICAMP}
\date{}

\begin{document}

\begin{frame}
  \titlepage
\end{frame}

% Uncomment these lines for an automatically generated outline.
%\begin{frame}{Outline}
%  \tableofcontents
%\end{frame}

\section{Introdução}

\begin{frame}{Revisão...}
  \Large
  \begin{enumerate}
  \item Máquinas de estado podem reconhecer categorias específicas de cadeias de
    caracteres,
  \item Expressões regulares são formas válidas de expressar máquinas de estados
    que reconhecem caracteres.
  \end{enumerate}
\end{frame}


\begin{frame}{Objetivos}
  \Large
  \begin{itemize}
    \item Entender o que é e como usar o Lex.
    \item Entender a sintaxe do Lex.
    \item Entender como compilar programas usando Lex.
    \item Construir uma pequena aplicação usando Lex.
    \item Entender o que é análise léxica.
  \end{itemize}
\end{frame}

\begin{frame}[fragile]{Exercício}
  \centering
  No código abaixo, identifique os liteirais inteiros e os números em ponto
  flutuante:

  \begin{verbatim}
  a = 50;
  b = 50.5 + 20;
  c = 60.2;
  \end{verbatim}

\end{frame}

\begin{frame}{Exercício}
  Ordene as instruções abaixo de forma a permitir reconhecer floats e inteiros
  num código:
  \begin{enumerate}
    \item Aplicar expressão regular \textsc{[0-9]+}. Caso sucesso...
    \item Aplicar expressão regular \textsc{[0-9]+[.][0-9]*}. Caso sucesso...
    \item Encontrei inteiro
    \item Encontrei float
    \item Caso contrário...
    \item Falha
  \end{enumerate}
\end{frame}

\begin{frame}[fragile]{Live Coding}
  \centering
  \large
  Lex é uma ferramenta que permite programar reconhecedores de tipos e criar
  programas que têm comportamentos distintos ao receber entradas de tipos
  diferentes.
  \begin{verbatim}

  %{
      /* Codigo C
  %}

     <----  definicoes

  %%
                 <--- Regras
  %%
  /* Codigo C */

  \end{verbatim}

\end{frame}


\begin{frame}[fragile]{Ambiguidades}
  \centering
  \large
  \begin{enumerate}
    \item Expressão que dá match com mais caracteres ganha
    \item Expressão que foi colocada mais acima ganha
  \end{enumerate}

  \begin{verbatim}
  [0-9]+ { printf("Expr1\n"); }
  [0-9A-Za-z]+ { printf("Expr2\n"); }
  \end{verbatim}

  Que entrada faria o programa escrever Expr1? E Expr2?
\end{frame}

\begin{frame}[fragile]{Acessando o texto que ``deu match''}
  \centering
  \large
  \textsc{char *yytext}

  \begin{verbatim}
  [0-9]+ { printf("%d\n", atoi(yytext)+1); }
  [a-zA-Z]+ { printf("%s\n", yytext); }
  \end{verbatim}
\end{frame}


\begin{frame}{Análise léxica de C}
\large
Mostre as expressões regulares capazes de reconhecer:
  \begin{enumerate}
    \item Caractere ponto-e-vírgula
    \item Palavras reservadas ``if'' e ``while''
    \item Números inteiros
    \item Números float
    \item Identificadores (começa com um caractere ascii seguido de qualquer
      número de caracteres alfanuméricos)
    \item Expressões com parênteses corretamente balanceados.
  \end{enumerate}
\end{frame}




\end{document}
